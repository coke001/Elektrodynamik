\begin{tabular}{L{2cm}lL{4cm}}
	\textbf{Symbol} &\textbf{Bezeichnung}              &\textbf{Einheit}\\
	$Q$             &elektrische Ladung                &\si{\A \s} = \si{C}  \\
	$\rho$          &elektrische Raumladungsdichte     &\si{\A \s / \m^3}  \\
	$\sigma$        &elektrische Flächenladungsdichte  &\si{\A \s / \m^2}  \\   
	$\tau$          &elektrische Linienladungsdichte   &\si{\A \s / \m}  \\
	$\varphi$       &elektrisches Potential            &\si{\V}  \\
	$U$             &elektrische Spannung              &\si{\V} = \si{J / (\A \s)} \\
	$\vec{E}$       &elektrische Feldstärke            &\si{\V / \m} = \si{\N / (\A \s)} \\
	$\Psi$          &elektrischer Fluss                &\si{\A \s} = \si{C} \\
	$\vec{D}$       &elektrische Flussdichte           &\si{\A \s / \m^2} \\
	$\vec{V}$       &elektrostatisches Vektorpotential &\si{\A \s / \m} \\
	$\vec{P}$       &elektrische Polarisation          &\si{\A \s / \m^2} \\
	$I$             &elektrische Stromstärke           &\si{\A} \\
	$\vec{J}$       &elektrische Stromdichte           &\si{\A / \m^2} \\
	$\vec{K}$       &elektrische Flächenstromdichte    &\si{\A / \m} \\
	$V$             &magnetische Spannung              &\si{\A} \\
	$\vec{H}$       &magnetische Feldstärke            &\si{\A / \m} \\
	$\Phi$          &magnetischer Fluss                &\si{\V \s} = \si{Wb}\\
	$\vec{B}$       &magnetische Flussdichte           &\si{\V \s / \m^2} = \si{T} \\
	$\vec{A}$       &magnetisches Vektorpotential      &\si{\V \s / \m} \\
	$\vec{M}$       &Magnetisierung                    &\si{\A / \m} \\
	$\varepsilon_0$ &elektrische Feldkonstante         &\si{\A \s / (\V \m)}    \\
	$\mu_0$         &magnetische Feldkonstante         &\si{\V \s / (\A \m)} = \si{\N / \A^2} \\
	$\vec{p}$       &elektrisches Moment               &\si{\A \s \m}    \\
	$\vec{m}$       &magnetisches Moment               &\si{\A \m^2}    \\
	$R$             &elektrischer Widerstand           &\si{\V / \A} = \si{\ohm} \\
	$L$             &elektrische Induktivität          &\si{\V \s / \A} = \si{H} \\
	$C$             &elektrische Kapazität             &\si{\A \s / \V} = \si{F} \\
	$\gamma$, $\sigma$  &elektrische Leitfähigkeit     &\si{\A / (\V \m)} = \si{S / \m}  \\
	$W(\mathcal{V})$    &Energieinhalt                 &\si{J} \\
	$w$             &Energiedichte                     &\si{J / \m^3} \\
	$Q(\mathcal{A})$     &Energiefluss                 &\si{J / \s} \\
	$\vec{q}$       &Energieflussdichte                &\si{J / (\s \m^2)} \\
	$r$             &(Volumens-)Leistungsdichte        &\si{W / \m^3} \\
	$r^s$           &Flächenleistungsdichte            &\si{W / \m^2} \\
	$\vec{G}(\mathcal{V})$       &Impulsinhalt         &\si{\kg \m / \s} \\
	$\vec{g}$       &Impulsdichte                      &\si{\kg \m / (\s \m^3)} \\
	$\vec{P}(\mathcal{A})$       &Impulsfluss          &\si{\kg \m / \s^2} \\
	$\uti{p}$       &Impulsflussdichte (Tensor 2. Stufe)            &\si{\kg \m / (\s^2 \m^2)} \\
	$\vec{F}$       &Kraft                             &\si{\kg \m / \s^2} = \si{\N} \\
	$\vec{f}$       &(Volumens-)Kraftdichte            &\si{\kg \m / (\s^2 \m^3)} = \si{\N / \m^3} \\
	$\vec{f}^s$     &Flächenkraftdichte                &\si{\kg \m / (\s^2 \m^2)} = \si{\N / \m^2} \\
	$\delta(\vec{r})$    &räumliche Dirac-Distribution              &\si{1 / \m^3}  \\
	$\vec{\nabla}$  &räumlicher Differentialoperator Nabla          &\si{1 / \m}  \\
	$\nabla^2 = \Delta$  &skalarer Diff.op. 2. Ord., Laplace-Op.    &\si{1 / \m^2}  \\
	$\vec{\kappa}$  &Ausbreitungsrichtung (reeller Einsvektor)      &\si{1} \\
	$\vec{n}$       &Einsnormalenvektor (auf Fläche)   &\si{1} \\
	$\vec{s}$       &Einstangentenvektor (an Kurve)    &\si{1} \\
	$\gamma$        &komplexer Ausbreitungskoeffizient &\si{1 / \m}  \\
	$\alpha$        &Dämpfungskoeffizient              &\si{1 / \m}  \\
	$\beta$         &Phasenkoeffizient                 &\si{1 / \m}  \\ 
\end{tabular}\\