\begin{question}[section=5,subsection=52,name={Energieflussdichte},difficulty=3,type=exercise,tags={20161207}]
	Berechnen Sie die zu einer ebenen Sinuswelle im leeren Raum mit der elektrischen Feldstärke (kartesischen Koordinaten) $\vec E(z,t) = \hat E \cos[2 \pi (t/T -z/\lambda)] \vec e_y$ gehörende, mittlere Energieflussdichte.
	\\ \textbf{Hinweis:}\\
	Wie sieht die zugehörige magnetische Feldstärke aus? Berechnen Sie den Poynting-Vektor und dessen Mittelwert.
\end{question}
\begin{solution}
	Der Poynting-Vektor ist durch $\vec S = \vec E \times \vec H$ definiert. Da wir uns im leeren Raum befinden, können wir uns die Berechnung der magnetischen Feldstärke sparen, da im leeren Raum auch folgende Beziehung gilt:
	\begin{align}
		\vec S &= \frac{E^2}{Z_0}\\
		\vec S &= \frac{\hat E ^2 \cdot \cos[ 2 \pi (t/T - z/\lambda)]^2}{\sqrt{\frac{\mu_0}{\varepsilon_0}}}\\
		\vec S &= \frac{\hat E ^2 \cdot (0,5 +  0,5 \cos[ 4 \pi (t/T - z/\lambda)])}{\sqrt{\frac{\mu_0}{\varepsilon_0}}}\\
		<\vec S> &= \frac{\hat E ^2 \cdot 0,5}{\sqrt{\frac{\mu_0}{\varepsilon_0}}}
	\end{align}
\end{solution}