\newpage % Begins the essay on a new page instead of on the same page as the table of contents 
	Werter Student!\\
	\\
	Diese Unterlagen werden dir \textbf{kostenlos} zur Verfügung gestellt, damit Sie dir im Studium behilflich sind.
	Sie wurden von vielen Studierenden zusammengetragen, digitalisiert und aufgearbeitet.
	Ohne der Arbeit von den Studierenden wären diese Unterlagen nicht entstanden und du müsstest dir jetzt alles selber zusammensuchen und von schlecht eingescannten oder abfotographierten Seiten lernen.
	Zu den Beispielen gibt es verschiedene Lösungen, welche du dir auch erst mühsamst raussuchen und überprüfen müsstest.
	Die Zeit die du in deine Suche und recherche investierst wäre für nachfolgende Studenten verloren. Diese Unterlagen leben von der Gemeinschaft die sie betreuen.
	Hilf auch du mit und erweitere diese Unterlagen mit deinem Wissen, damit sie auch von nachfolgenden Studierenden genutzt werden können.
	Geh dazu bitte auf \href{https://github.com/Painkilla/VU-351.019-Elektrodynamik/issues}{https://github.com/Painkilla/VU-351.019-Elektrodynamik/issues} und schau dir in der TODO Liste an was du beitragen möchtest.
	Selbst das Ausbessern von Tippfehlern oder Rechtschreibung ist ein wertvoller Beitrag für das Projekt. Nütze auch die Möglichkeit zur Einsichtnahme von Prüfungen zu gehen und die Angaben anderen zur Verfügung zu stellen, damit die Qualität der Unterlagen stetig besser wird.
	\href{https://www.latex-project.org/get/}{\LaTeX} und \href{https://git-scm.com/downloads}{Git} sind nicht schwer zu lernen und haben auch einen Mehrwert für das Studium und das spätere Berufsleben.
	Sämtliche Seminar oder Bachelorarbeiten sind mit \href{https://www.latex-project.org/get/}{\LaTeX} zu schreiben. \href{https://git-scm.com/downloads}{Git} ist ideal um gemeinsam an einem Projekt zu arbeiten und es voran zu bringen.
	Als Student kann man auf GitHub übrigens kostenlos unbegrenzt private Projekte hosten.\\
	Mit dem Befehl:\\
	\texttt{\$ git clone https://github.com/Painkilla/VU-351.019-Elektrodynamik.git}\\
	erstellst du eine lokale Kopie des Repositorium. Du kannst dann die Dateien mit einem \href{https://www.latex-project.org/get/}{\LaTeX-Editor} deiner Wahl bearbeiten und dir das Ergebniss ansehen.
	Bist du auf GitHub regestriert, kannst du einen Fork(engl:Ableger) erstellen und mit den Befehlen:\\
	\texttt{\$ git commit -m ``Dein Kommentar zu den Änderungen''}\\
	\texttt{\$ git push}\\
	werden deine Ergänzungen auf deinen Ableger am Server gesendet. Damit deine Ergänzungen auch in das zentrale Repositorium gelangen und allen Studierenden zur Verfügung steht musst du nur noch einen Pull-Request erstellen.
	\newpage